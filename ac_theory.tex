\documentclass{article}
\usepackage[utf8]{inputenc} %sets input encoding to UTF-8, needed for Polish, Japanese, etc.
\usepackage[T1]{fontenc} %needed for Polish characters
\usepackage{lmodern} %this font handles Polish characters properly
\usepackage[cm]{fullpage} %very small margins (around 1.5cm)
\usepackage{multicol} %use \begin{multicols}{#} for # columns
\usepackage{amssymb} %big bracets and other useful symbols
\usepackage{amsmath} %some mathematical symbols
\usepackage{enumitem} %remove vertical space in itemize with: [noitemsep,nolistsep]

\newcommand{\separator}{\noindent \line(1,0){529}}

\begin{document}

\title{Theory for \emph{Algorithms \& Computability} course on MiNI on WUT}
\date{\today}
\author{Mateusz Bysiek, Computer Science, MiNI, WUT}
\maketitle

\tableofcontents

\pagebreak[4]

%\section{A\&C Theory}

\section{Introductory test}

%\subsubsection{Problem, instance, algorithm}

\begin{multicols}{2}

\paragraph{Problem}
cannot be defined exactly. In most broad meaning it is something unknown for which an answer is
needed, and such answer can be sometimes given in form of an algorithm. Problem can be defined as a
set of questions of similar kind. Problem is a set of instances.

\paragraph{Decision problem} is a problem, for which the answer is either ``Yes'' or ``No''.

\paragraph{Optimization problem} is to find solutions and choose the most optimal (the best, ideal,
etc.) one.

\paragraph{Instance of a problem}
Is a concrete question from a set of questions that make a full problem.

\paragraph{Countability of a set of instances}
\ldots the set of instances of a problem can sometimes be shown to be countable, for example in case
of sorting problem for the set of natural numbers.

\paragraph{Algorithm}
solves all instances of the problem. It is a description of how to solve instances of a problem,
based on very basic operations. It is a sequence of instructions.

\end{multicols}

\separator

%\subsubsection{Operations}

Having a problem $\Pi$, set of instances of that problem $D_\Pi$, and instance $d$ of a problem
$\Pi$, i.e. $d \in D_\Pi$ \ldots

\begin{multicols}{2}

\paragraph{Elementary operation}
is an indivisible operation on two basic data cells. Depending on cost criteria used, it takes unit
or logarithmic time.

%\end{multicols}

%\separator

%\begin{multicols}{2}

\paragraph{Time cost function}
$C_t(d) = $ number of elementary operations' executions for the instance $d$.
\begin{gather*}
C_t : D_\Pi \rightarrow N
\end{gather*}

Cost function is a function that goes from set of instances of problem to natural numbers. The value
of cost time function is a number of executions of elementary operations needed to solve given
instance of a problem.

\paragraph{Space cost function}
$C_s(d) = $ number of cells of memory used to solve the instance $d$.
\begin{gather*}
C_s : D_\Pi \rightarrow N
\end{gather*}

\paragraph{Dominant operations}
of algorithm $A$ is a set of operations that are performed 
if any other operation of $A$ is performed.

In other words, $\exists k_1, k_2: k_1 d_t(d) \leq C_t(d) \leq k_2 d_t(d)$, where $C_t(d)$ is a
value of cost function for $d$, and $d_t(d)$ is number of dominant operations' executions for the
same instance.

\end{multicols}

\separator

%\subsubsection{Complexity}
\begin{multicols}{2}

\paragraph{Pessimistic time complexity}
is the maximum possible result of the time cost function for a given size of input $n$, 
therefore time complexity is a function of $n$.
\begin{gather*}
T_p : | D_\Pi | \rightarrow N \\
T_p(n) = \sup \left\{ C_t(d) : d \in D_\Pi \wedge |d| = n \right\}
\end{gather*}

\paragraph{Average time complexity}
\begin{gather*}
T_a : | D_\Pi | \rightarrow R_+ \\
T_a(n) = \sum_{d \in D_\Pi, |d|=n} C_t(d) * pr(d)
\end{gather*}

Where $pr(d)$ is a probability of encountering a specific instance $d$. In case of average
complexity the set of all instances $D_\Pi$ is finite.

\paragraph{Pessimistic space complexity}
is the maximum possible result of the space cost function for a given size of input $n$, 
therefore space complexity is a function of $n$.
\begin{gather*}
S_p : | D_\Pi | \rightarrow N \\
S_p(n) = \sup \left\{  C_s(d) : d \in D_\Pi \wedge |d| = n \right\}
\end{gather*}

\end{multicols}

\separator

%\subsubsection{Cost criteria}

\begin{multicols}{2}

\paragraph{Uniform cost criteria}
Elementary operation needs constant time $t = 1$, ``unit time''.

\paragraph{Logarithmic cost criteria}
Sum of lengths of arguments (or logarithms of their values) is the time needed to perform
an elementary operation.

\end{multicols}

\pagebreak[4]

\section{Test \#1}

\subsection{RAM - random access memory machines}

\begin{itemize}[noitemsep,nolistsep]
  \item Labels: LABEL
  \item Functions: add, del, clr, $\leftarrow$, jmp, jmp$n$ ($n \in \Sigma$), continue
\end{itemize}

%Example implementations of few algorithms for RAM machines

% \begin{multicols}{2}
% 
% \begin{minted}[mathescape,linenos,numbersep=5pt,gobble=0,framesep=2mm,xleftmargin=5pt]{c++}
%    clr R
%    clr Q
%    P jmp0 ZERO
%    P jmp1 ONE
%    COPY_RESULT
% ZERO
%    del P
%    jmp WR_Q_0
% ONE
%    del P
%    jmp WR_Q_1
% WR_R_0
% WR_R_1
% WR_Q_0
% WR_Q_1
% \end{minted}
% \begin{minted}[mathescape,linenos,numbersep=5pt,gobble=0,framesep=2mm,xleftmargin=5pt]{c++}
% MOVE_Q_TO_R
% MOVE_R_TO_Q
% COPY_RESULT
%    Q jmp0 COPY_Q
%    Q jmp1 COPY_Q
%    P $\leftarrow$ R
%    jmp END
% COPY_Q
%    P $\leftarrow$ Q
%    jmp END
% END
%    clr R
%    clr Q
%    continue
% \end{minted}
% \end{multicols}

\subsection{PRF - primitive recursive functions}
Proofs that given functions are primitive recursive.

\paragraph{Addition} $+(x,y) = x+y$
\begin{itemize}[noitemsep,nolistsep]
  \item $+(x,0) = x = g(x) = P_{1,1}(x)$
  \item $g = P_{1,1}$
  \item $+(x,y+1) = +(x,y)+1 = h(x,y,+(x,y)) = S(P_{3,3}(x,y,+(x,y)))$
  \item $h = S \circ P_{3,3}$
\end{itemize}

\paragraph{Multiplication} $*(x,y) = x*y$
\begin{itemize}[noitemsep,nolistsep]
  \item $*(x,0) = x = g(x) = P_{1,1}(x)$
  \item $g = P_{1,1}$
  \item $*(x,y+1) = *(x,y)+x = h(x,y,+(x,y)) = +(P_{3,3}(x,y,+(x,y)),P_{3,1}(x,y,+(x,y)))$
  \item $h = + \circ (P_{3,3},P_{3,1})$
\end{itemize}

\paragraph{Factorial} $!(x) = x!$
\begin{itemize}[noitemsep,nolistsep]
  \item $!(0) = 1 = g() = S(Z_0())$
  \item $g = S \circ Z_0$
  \item $!(x+1) = !(x)*(x+1) = g(x,!(x)) = *(P_{2,2}(x,!(x)),S(P_{2,1}(x,!(x))))$
  \item $g = * \circ (P_{2,2}, S \circ P_{2,1})$
\end{itemize}

\paragraph{Exponentiation} $exp(x,y) = x^y$
\begin{itemize}[noitemsep,nolistsep]
  \item $exp(x,0) = 1 = g(x) = S(Z_1(x))$
  \item $g = S \circ Z_1$
  \item $exp(x,y+1) = exp(x,y)*x = g(x,y,exp(x,y)) = *(P_{3,3}(x,y,exp(x,y)),P_{3,1}(x,y,exp(x,y)))$
  \item $g = * \circ (P_{3,3}, P_{3,1})$
\end{itemize}

\paragraph{Predecessor} $pred(x) = 
\begin{cases}
x-1 \mbox{ if } x>0 \\
0 \mbox{ otherwise}
\end{cases}$
\begin{itemize}[noitemsep,nolistsep]
  \item $pred(0) = 0 = g() = Z_0()$
  \item $g = Z_0$
  \item $pred(x+1) = x = h(x) = P_{1,1}(x)$
  \item $h = P_{1,1}$
\end{itemize}

\paragraph{Limited difference} $ld(x,y) = 
\begin{cases}
x-y \mbox{ if } x>y \\
0 \mbox{ otherwise}
\end{cases}$
\begin{itemize}[noitemsep,nolistsep]
  \item $ld(x,0) = x = g(x) = P_{1,1}(x)$
  \item $g = P_{1,1}$
  \item $ld(x,y+1) = ld(x,y)-1 = h(x,y,ld(x,y)) = pred(P_{3,3}(x,y,ld(x,y)))$
  \item $h = pred \circ P_{3,3}$
\end{itemize}

\paragraph{Sign} $sg(x) = \begin{cases}
0 \mbox{ if } x=0 \\
1 \mbox{ otherwise}
\end{cases}$
\begin{itemize}[noitemsep,nolistsep]
  \item $sg(0) = 0 = g() = Z_0()$
  \item $g = Z_0$
  \item $sg(x+1) = 1 = h(x) = S(Z_1(P_{1,1}(x)))$
  \item $h = S \circ Z_1 \circ P_{1,1}$
\end{itemize}

% \paragraph{Comparison} $gt(x,y)$
% 
% $lt(x,y)$ 

\paragraph{Quotient} $quo(x,y) = 
\begin{cases}
\frac{y}{x} \mbox{ if } x>0 \\
0 \mbox{ otherwise}
\end{cases}$
\begin{itemize}[noitemsep,nolistsep]
  \item $quo(x,y) = \mu_{0 \leq i \leq y} \left[ gt(*(x,S(z)),y) \right] $
  %\item $g = Z_1$
  %\item $ $
  %\item $ $
\end{itemize}

% \paragraph{Bounded sum} 

% \paragraph{Bounded product} 

\paragraph{Bounded minimum} $f(x_1,\ldots,x_n,y) = 
\mu_{0 \leq i \leq y} \left[ k(x_1,\ldots,x_n,i) = 0 \right] = $

$\begin{cases}
\mbox{minimal value, for which } k(x_1,\ldots,x_n,i) = 0 \\
0 \mbox{ if such } i \mbox{ doesn't exist}
\end{cases}$
% \begin{itemize}[noitemsep,nolistsep]
%   \item $quo(x,y) = \mu_{0 \leq i \leq y} \left[ gt(*(x,S(z)),y) \right] $
% \end{itemize}

% \paragraph{XOR} 
% \begin{itemize}[noitemsep,nolistsep]
%   \item 
%   \item 
%   \item 
%   \item 
% \end{itemize}

% \paragraph{Equality test} 
% \begin{itemize}[noitemsep,nolistsep]
%   \item 
%   \item 
%   \item 
%   \item 
% \end{itemize}

% \paragraph{Primality test} 

% \paragraph{Divisibility test} 
% \begin{itemize}[noitemsep,nolistsep]
%   \item 
%   \item 
%   \item 
%   \item 
% \end{itemize}

\pagebreak[4]

\subsection{A(x,y) - Ackerman function}
\begin{gather*}
A(x,y) = \begin{cases}
A(0,y) = y+1 \\
A(x+1,0) = A(x,1) \\
A(x+1,y+1) = A(x,A(x+1,y))
\end{cases}
\end{gather*}
\noindent \textbf{3 basic properties}:
\begin{enumerate}[noitemsep,nolistsep]
  \item it is increasing
  \item it has values greater than the arguments
  \item first parameter affacts the result much more than the second one, i.e. $A(x+2,y) > A(x,2y) $
\end{enumerate}

\subsubsection{Ackerman funcion is not a PRF}
\textbf{Theorem}: Ackerman funcion is not a PRF, because:
\begin{gather*}
\forall(f(x_1,\ldots,x_n))\exists k \mbox{ such that } A(k,\max\{x_1,\ldots,x_n\}) > f(x_1,\ldots,x_n)
\end{gather*}
where $f$ is any primitive recursive function of $n$ arguments.

\textbf{Proof} by induction
\begin{enumerate}
  \item We show that such constant exists for all 3 basic PRFs.
  \begin{enumerate}[noitemsep,nolistsep]
    \item $A(0,\max\{x_1,\ldots,x_n\}) > Z_n(x_1,\ldots,x_n)$
    \item $A(1,x) > S(x)$
    \item $A(0,\max\{x_1,\ldots,x_n\}) > x = P_{n,i}(x_1,\ldots,x_n)$ where $i$ is index 
    of argument with maximum value.
   \end{enumerate}
  \item We show that such constant exists for all compositions of those functions, and, therefore 
  for any PRF.
  
  %emph
  \textbf{Lemma 1}
  
  If for PRF $g,f_1,\ldots,f_n$ such that $g: \mathbb{N}^m \rightarrow \mathbb{N}$ and 
  $f_{1 \ldots n}: \mathbb{N}^n\rightarrow \mathbb{N}$ we have constants defined in the theorem, 
  then there is a constant $k$ such that:
  
  $A(k,\max\{x_1,\ldots,x_n\}) > [g \circ (f_1,\ldots,f_n)](x_1,\ldots,x_n)$ 
  
  \textbf{Lemma 2} 
  
  If for PRF $g,h$ such that $g: \mathbb{N}^n \rightarrow \mathbb{N}$ and 
  $h: \mathbb{N}^{n+2} \rightarrow \mathbb{N}$ we have constants defined in the theorem,
  then there is a constant $k$ such that:
  
  $A(k,\max\{x_1,\ldots,x_{n+1}\}) > f(x_1,\ldots,x_{n+1})$ where $f$ is obtained by PRS applied to
  $g$ and $h$.
\end{enumerate}

\subsubsection{Ackerman function is recursive}
\textbf{Theorem}: Ackerman function is recursive, because we can construct it using PRS and
unbounded minimum.

When we evaluate for example $A(2,1)$:
\begin{gather*}
A(2,1) = A(1,A(2,0)) = A(1,A(1,1)) = A(1,A(0,A(1,0))) = \ldots = A(0,4) = 5
\end{gather*}
We can convert the notation by omitting the $A$'s to the following:
$(2,1) = (1,2,0) = (1,1,1) = (1,0,1,0) = \ldots = (0,4) = 5$ \\
and then encode it using Gödel Encoding: $2^2 3^2 5^1 = 2^3 3^1 5^2 7^0 = 2^3 3^1 5^1 7^1 
= 2^4 3^1 5^0 7^1 11^0 = \ldots = 2^2 3^0 5^4 = 2^1 3^5$

Then, we may define the Trace function $\psi(x,y,n)$ being the Gödel number of $n$-th step of
computation of $A(x,y)$. Then, we define $n_0$ as a step, in which we reach the final result.

Full definition:
\begin{gather*}
\psi(x,y,n) = \begin{cases}
\mbox{Gödel number of } n\mbox{-th step of computation of } A(x,y) \\
\psi(x,y,n_0) \mbox{ for } n > n_0
\end{cases}
\end{gather*}

$E(n,z)$ is the exponent of Gödel number $z$ at its $n$-th place. Example: $E(0,2^2 3^5 5^1) =
E(0,(2,5,1)) = 2$ means that exponent of 2 (being 0th term of Gödel-encoded number) is equal to 2.
$E(1,2^2 3^4 5^3) = E(1,(2,4,3)) = 4$ means that exponent of 3 (being 1st term of Gödel-encoded
number) is equal 4.
\begin{gather*}
\eta(x,y) = n_0 = \mu_{n \geq 0} \left[ L(\psi(x,y,n)) = 1 \right]
\end{gather*}
\begin{gather*}
A(x,y) = E(1,\psi(x,y,\eta(x,y)))
\end{gather*}

\pagebreak[4]

\section{Test \#2}

Sorry, this section will probably not be updated, because I already passed this subject ;)

\subsection{Recursive functions}

\subsection{Classes of languages, problems, functions}

\paragraph{Universal language}

\paragraph{Diagonal language}

\paragraph{Equivalence of groups of classes of languages, problems, functions}

\subsection{Cook's theorem}

\subsection{P vs. NP}


\end{document}
